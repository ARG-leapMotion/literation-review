%\documentclass[11pt, oneside]{article}										% use "amsart" instead of "article" for AMSLaTeX format
%\usepackage{geometry}													    % See geometry.pdf to learn the layout options. There are lots.
%\geometry{a4paper}													        % ... or a4paper or a5paper or ... 
%\geometry{landscape}													    % Activate for rotated page geometry
%\usepackage[parfill]{parskip}												% Activate to begin paragraphs with an empty line rather than an indent
%\usepackage{graphicx}                                                      % Use pdf, png, jpg, or eps§ with pdflatex; use eps in DVI mode		
%\usepackage{amssymb}
%SetFonts
%SetFonts
%\date{}																    % Activate to display a given date or no date
%--------------------------------------------------------------------------------------------------------------------------------------------

\documentclass[12pt, oneside]{article}										% use "amsart" instead of "article" for AMSLaTeX format
\usepackage[showframe=true]{geometry}													    % See geometry.pdf to learn the layout options. There are lots.
\geometry{a4paper}														    % ... or a4paper or a5paper or ... 
\geometry{left=1.5cm,right=1.5cm,top=1cm,bottom=2cm}                        % Set margin of the paper
\usepackage{graphicx}													    % Use pdf, png, jpg, or eps§ with pdflatex; use eps in DVI mode
\usepackage{titling}
\usepackage{blindtext}                                                      % Dummy Text
\usepackage{multicol}                                                       % Multi column pack


\title{Animal Racing Video Game Using Hand Motion}
\author{Thanut Sajjakulnukit, Nutcharueta Sihirunwong, Budnampetch Onmee}

\begin{document}
    \maketitle														        % Title and author
    \begin{multicols*}{2}
    \twocolumn
    \begin{abstract}                                                        % Start Abstract here
        Nowadays, video games are among the best sources of 
        entertainment that people use to enjoy themselves together.
            In addition to the entertainment, the player will also gain 
            knowledge or skills depending on the purpose of the game’s 
            creators.   LEAP motion technology is used in order to 
            present new ways for players to interact with our video 
            game, as well as provide more satisfying experiences for 
            everybody involved. Moreover, leap motion can increase 
            challenging in game and also help players are really into 
            controlling character in game. We  develop a new kind of 
            game which uses your own hands to control the character 
            movements via different gestures. This project is not 
            nly for entertainment, but also for the illustrations 
            of animal mechanisms so that kids can learn. This 
            project also lets people participate in some 
            activities together.

    \end{abstract}

    \section{Introduction}                                                  % Start Introduction here
    Currently, the social condition makes a pressure on people’s 
    stress. Video games are one of the answers to help relieve 
    stress. Today there are many developers who created the video 
    game in the market but it doesn't have a new technology to make 
    a different video game. Leap Motion is one of technology that 
    can be used with the game. It is hand tracking technology is 
    designed to be embedded directly into VR/AR headset. \\

    In this project, we will use Leap Motion technology, one of
    the new technology that use virtual hand to control object
    in video game. To create a racing video game by tracking 
    hand to control character’s movement depending on the 
    physical characteristic of each animal. The movement of 
    character in the game is related to movement of the hand 
    which makes the game more challenge and interesting.
    
        
    \section{Literature Summary}                                            % Start literature summary here

    The research paper and publish experiment are related to video game 
    technology and creating process which is the technology that we 
    focussing in this project. First, the experiment which is working 
    on the variability of the sensor. Then the following is research 
    about detect hand motion and identify hand gesture[1] based on 
    the evaluation of the temporal variability of the strokes. It 
    shows some differences between the stability of the different 
    measurements which can be produced either by the user or the 
    sensor. \\

    Second, An Anthropomorphic Hand with Five Fingers 
    Controlled by a Motion Leap Device research[2] test leap motion 
    for capturing the movements of a human hand's fingers in different 
    configurations. Then create software to facilitate interaction 
    between a human hand. It conclude that object recognition has 
    a high accuracy rate where contrast between the object and its 
    background is high. \\
    
    Third, “Evaluate of natural user interface 
    usability based on the Leap Motion device” focus on users expressed 
    disappointment with the device. Some user state that, for the 
    experience in fact be "natural", it is necessary that the user has 
    the feeling of always being in control, with minimal time and 
    effort as possible. \\
    
    Fourth, the “Dynamic Simulations in a 3D-Environment 
    a comparison between Maya and Blender” research [4] is focussing on 
    software tools for creating 3D modeling by comparing process of two 
    software tools. So In our project we choose blender for creating model 
    on the game because it is user friendly like unity3D so we might not have 
    to research from the beginner. \\
    
    Last, the article describes making 
    difficult fun in the video game to make a player enjoy and make difficult 
    with other video game [5].It show making a game challenge include a choice 
    of difficulties easy or hard can make everyone can enjoy a video game. 
    There exists the possibility to design a video game for all skill 
    levels: where a crossover of players can enjoy the game without 
    being forced into a particular path.

    
    \section{Reference}
        \begin{itemize}
            \item[\textbf{[1]}] Elyoenai Guerra-Segura,Carlos M. Travieso and Jesús B. Alonso, “Study of the 
            variability of the Leap Motion’s measures for its use to characterize air strokes”, 
            in Measurement, Vol. 105, pp. 87-97, July 2017

            \item[\textbf{[2]}] Catalin Constantin Moldovan and Ionel Staretu, “An Anthropomorphic Hand with Five 
            Fingers Controlled by a Motion Leap Device”, in Procedia Engineering, Vol. 181, 
            pp. 575-582, 2017
            

            \item[\textbf{[3]}] Christianne Falcao,Ana Catarina Lemos and Marcelo Soares, “Evaluation of Natural User 
            Interface: A Usability Study Based on the Leap Motion Device”, 
            in Procedia Manufacturing, Vol. 3, pp. 5490-5495, 2015

            \item[\textbf{[4]}] Dario Senkic, “Dynamic Simulations in a 3D-Environment a comparison between Maya
            and Blender”, in Academy for technology and environment University of Gävle, 
            September 2010
            

            \item[\textbf{[5]}] Darran Jamieson, “Making Difficult Fun: How to Challenge Your Players”, [Online]
            Available: https://gamedevelopment.
            tutsplus.com/tutorials/making-difficult-fun-how-to-challenge-your-players--cms-25873 [Accessed: May 20, 2016]
            
        \end{itemize}

    \end{multicols*}
\end{document}