%\documentclass[11pt, oneside]{article}										% use "amsart" instead of "article" for AMSLaTeX format
%\usepackage{geometry}													    % See geometry.pdf to learn the layout options. There are lots.
%\geometry{a4paper}													        % ... or a4paper or a5paper or ... 
%\geometry{landscape}													    % Activate for rotated page geometry
%\usepackage[parfill]{parskip}												% Activate to begin paragraphs with an empty line rather than an indent
%\usepackage{graphicx}                                                      % Use pdf, png, jpg, or eps§ with pdflatex; use eps in DVI mode		
%\usepackage{amssymb}
%SetFonts
%SetFonts
%\date{}																    % Activate to display a given date or no date
%--------------------------------------------------------------------------------------------------------------------------------------------

\documentclass[journal]{IEEEtran}										    % use "amsart" instead of "article" for AMSLaTeX format
\usepackage[showframe=true]{geometry}										% See geometry.pdf to learn the layout options. There are lots.
\geometry{left=1.7cm,right=1.7cm,top=1.9cm,bottom=2cm}                      % Set margin of the paper
\usepackage[pdftex]{graphicx}													    % Use pdf, png, jpg, or eps§ with pdflatex; use eps in DVI mode
\graphicspath{img/}
% \usepackage{blindtext}                                                      % Dummy Text



\title{\textbf{Animal Racing Video Game Using Hand Motion}}
\author{Thanut Sajjakulnukit, Nutcharueta Sihirunwong, Budnampetch Onmee \\
    \Large Kasetsart University }

\begin{document}
    \markboth{Journal of \LaTeX\ Class Files,~Vol.~1, No.~8, April~2018}
    {Shell \MakeLowercase{\textit{et al.}}: Bare Demo of IEEEtran.cls for IEEE Journals}
    \maketitle
    
    \begin{abstract}                                                        % Start Abstract here
        Nowadays, video games are among the best sources of entertainment 
        that people use to enjoy themselves together. In addition to the 
        entertainment, the player will also gain knowledge or skills 
        depending on the purpose of the game’s creators. We  develop a new 
        kind of game that uses Leap Motion to detect your own hands and 
        control the character movements via different gestures. 
        The Leap Motion will make game more fun, challenge, and interesting. 
        This project is not only for entertainment, but also for the 
        illustrations of animal mechanisms so that kids can learn. 
        This project also lets people participate in some activities together.

    \end{abstract}

    \section{Introduction}                                                  % Start Introduction here
        \IEEEPARstart{C}{urrently}, the social condition makes a pressure 
        on people’s stress. Video games are one of the answers to help relieve 
        stress. Today there are many developers who created the video 
        game in the market but it doesn't have a new technology to make 
        a different video game. Leap Motion is one of technology that 
        can be used with the game. It is hand tracking technology is 
        designed to be embedded directly into VR/AR headset. 

        In this project, we will use Leap Motion technology, one of
        the new technology that use virtual hand to control object
        in video game. To create a racing video game by tracking 
        hand to control character’s movement depending on the 
        physical characteristic of each animal. The movement of 
        character in the game is related to movement of the hand 
        which makes the game more challenge and interesting.
        
    \section{Literature Summary}                                            % Start literature summary here
        Literation text here...


    \section{Reference}
        \begin{itemize}
            \item[\textbf{[1]}] Elyoenai Guerra-Segura,Carlos M. Travieso and Jesús B. Alonso, “Study of the 
            variability of the Leap Motion’s measures for its use to characterize air strokes”, 
            in Measurement, Vol. 105, pp. 87-97, July 2017

            \item[\textbf{[2]}] Catalin Constantin Moldovan and Ionel Staretu, “An Anthropomorphic Hand with Five 
            Fingers Controlled by a Motion Leap Device”, in Procedia Engineering, Vol. 181, 
            pp. 575-582, 2017
            

            \item[\textbf{[3]}] Christianne Falcao,Ana Catarina Lemos and Marcelo Soares, “Evaluation of Natural User 
            Interface: A Usability Study Based on the Leap Motion Device”, 
            in Procedia Manufacturing, Vol. 3, pp. 5490-5495, 2015

            \item[\textbf{[4]}] Dario Senkic, “Dynamic Simulations in a 3D-Environment a comparison between Maya
            and Blender”, in Academy for technology and environment University of Gävle, 
            September 2010
            

            \item[\textbf{[5]}] Darran Jamieson, “Making Difficult Fun: How to Challenge Your Players”, [Online]
            Available: https://gamedevelopment.
            tutsplus.com/tutorials/making-difficult-fun-how-to-challenge-your-players--cms-25873 [Accessed: May 20, 2016]
            
        \end{itemize}


\end{document}